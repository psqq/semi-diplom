\chapter{Графы и их инварианты}

\section{Определения}

\begin{my_def}
Граф, или неориентированный граф $G$ --- это упорядоченная пара~$G=(V,E)$,
где $V$ --- это не пустое множество вершин или узлов, а $E$ --- множество пар
(в случае неориентированного графа — неупорядоченных) вершин, называемых ребрами.
\end{my_def}

\begin{my_def}
Ориентированный граф (сокращенно орграф)  $G$ --- это упорядоченная пара~$G=(V,A)$,
где  $V$ — не пустое множество вершин или узлов, и  $A$ — множество (упорядоченных)
пар различных вершин, называемых дугами или ориентированными ребрами.
\end{my_def}

Определения пути, цикла, цепи и связанные с ними:

\begin{my_def}
Пусть $G$ --- неориентированный граф. Путём в $G$ называется такая конечная или
бесконечная последовательность ребер и вершин
$$S = (..., a_0,E_0, a_1, E_1, ..., E_{n-1}, a_n, ...),$$
что каждые два соседних ребра  $E_{i-1}$ и  $E_{i}$ имеют общую вершину  $a_{i}$.
Таким образом, можно написать
$$..., E_0=(a_0,a_1), E_1=(a_1,a_2), ... , E_n=(a_n,a_{n+1}), ...$$
Отметим, что одно и тоже ребро может встречаться в пути несколько раз. Если нет ребер,
предшествующих  $E_{0}$, то  $a_{0}$ называется начальной вершиной, а если нет
ребер, следующих за  $E_{(n-1)}$, то  $a_n$ называется конечной вершиной.
Любая вершина, принадлежащая двум соседним рёбрам, называется внутренней.
Так как рёбра и вершины в пути могут повторяться, внутренняя вершина может оказаться
начальной или конечной вершиной. Если начальная и конечная вершины совпадают,
путь называется циклическим. Путь называется цепью, а циклический путь — циклом,
если каждое его ребро встречается не более одного раза (вершины могут повторяться).
Не циклическая цепь называется простой цепью, если в ней никакая вершина не повторяется.
Цикл с концом  $a_{0}$ называется простым циклом, если  $a_{0}$ не является в нём
промежуточной вершиной и никакие другие вершины не повторяются.
\end{my_def}

\begin{my_def}
Связный граф --- граф, между любой парой вершин которого существует как
минимум один путь.
\end{my_def}

\begin{my_def}
Компонента связности графа  $G$ --- максимальный связный подграф графа  $G$.
\end{my_def}

Матрица смежности --- это один из способов представления графа:

\begin{my_def}
Матрицей смежности $A=||\alpha_{i,j}||$ графа $G=(V,E)$ называется матрица $A_{[V\times{}V]}$, в которой $\alpha_{i,j}$ --- количество рёбер,
соединяющих вершины $v_i$ и $v_j$, причём при $i=j$ каждую петлю учитываем дважды,
если граф не является ориентированным, и один раз, если граф ориентирован.
\end{my_def}

\begin{my_def}
Кликой неориентированного графа называется подмножество его вершин, любые две из
которых соединены ребром.

Максимальная клика --- это клика, которая не может быть
расширена путём включения дополнительных смежных вершин, то есть нет клики большего
размера, включающей все вершины данной клики.
\end{my_def}

\begin{my_def}
Если $v_{1}, v_{2}$ --- вершины, а  $e=(v_{1},v_{2})$ --- соединяющее их ребро, то
говорят, что вершины $v_{1}, v_{2}$ инцидентные ребру $e$.
\end{my_def}

\begin{my_def}
Степень вершины в теории графов --- количество рёбер графа $G$, инцидентных вершине $x$.
Обозначается $deg(x)$.
\end{my_def}

\begin{my_def}
Граф называется однородным (регулярным), если степени всех его вершин равны.
Если степени вершин однородного графа равны $k$, то граф называют $k$-однородным.
\end{my_def}

\begin{my_def}
Изоморфизмом графов  $G=\left(V_{G},E_{G}\right)$  и
$H=\left(V_{H},E_{H}\right)$  называется биекция между множествами
вершин графов  $f\colon \ V_{G}\rightarrow V_{H}$ такая, что любые две вершины
$u$ и  $v$ графа  $G$ смежны тогда и только тогда, когда вершины  $f(u)$ и $f(v)$
смежны в графе  $H$.
\end{my_def}

\begin{my_def}
Дерево --- связный граф, не содержащий циклов.
\end{my_def}

\begin{my_def}
Корень дерева --- выбранная вершина дерева; в орграфе --- вершина с нулевой степенью захода.
\end{my_def}

\section{Примеры инвариантов}

\begin{my_def}
Инвариант графа --- некоторое обычно числовое значение или упорядоченный
набор значений, одинаковый для изоморфных графов.
\end{my_def}

\begin{itemize}
\item Диаметр графа  $diam(G)$ --- длина кратчайшего пути (расстояние) между парой наиболее удаленных вершин.
\item Индекс Винера — величина  $w=\sum _{{\forall i,j}}d(v_{i},v_{j})$, где  $d(v_{i},v_{j})$ — минимальное расстояние между вершинами  $v_{i}$ и  $v_{j}$.
\item Индекс Рандича — величина  $r=\sum _{{\forall i,j}}{\frac  {1}{{\sqrt  {d(v_{i})d(v_{j})}}}}$.
\item Минимальное число вершин, которое необходимо удалить для получения несвязного графа.
\item Минимальное число ребер, которое необходимо удалить для получения несвязного графа.
\item Обхват графа — число ребер в составе минимального цикла.
\item Определитель матрицы смежности.
\item Плотность графа  $\varphi (G)$ --- число вершин максимальной по включению клики.
\item Упорядоченный по возрастанию или убыванию вектор степеней вершин  $s(G)=(d(v_{1}),d(v_{2}),\dots ,d(v_{n}))$ --- при
использовании переборных алгоритмов определения изоморфизма графов в качестве
предположительно-изоморфных пар вершин выбираются вершины с совпадающими степенями,
что способствует снижению трудоемкость перебора. Использование данного инварианта для
$k$-однородных графов не приводит к снижению вычислительной сложность перебора, так как
степени всех вершин подобного графа совпадают:  $s(G)=(k,k,\dots ,k)$.
\item Упорядоченный по возрастанию или убыванию вектор собственных чисел матрицы смежности графа (спектр графа). Сама по себе матрица смежности не является инвариантом, так как при смене нумерации вершин она претерпевает перестановку строк и столбцов.
\item Число вершин  $n(G)=|A|$ и число дуг/ребер  $m(G)=|V|$.
\item Число компонент связности графа  $\kappa (G)$.
\item Характеристический многочлен матрицы смежности.
\item Хроматическое число  $\chi (G)$ --- минимальное число цветов, в которые можно
раскрасить вершины графа $G$ так, чтобы концы любого ребра имели разные цвета.
\end{itemize}
