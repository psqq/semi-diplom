\begin{verbatim}
     1	#include "digraph.h"
     2	#include <bits/stdc++.h>
     3	using namespace std;
     4	
     5	digraph::digraph(istream &is) { load(is); }
     6	
     7	digraph::digraph(string s) {
     8	  stringstream ss(s);
     9	  load(ss);
    10	}
    11	
    12	digraph::digraph(int an) { resize(an); }
    13	
    14	void digraph::resize(int new_n) {
    15	  n = new_n;
    16	  adj_list.resize(n, vector<int>());
    17	  input_vers.resize(n, vector<int>());
    18	  adj_mat.resize(n, vector<int>(n, 0));
    19	  dist.resize(n, vector<int>(n, -1));
    20	  inv3.resize(n);
    21	}
    22	
    23	void digraph::load_from_file(string file_name) {
    24	  ifstream fin(file_name);
    25	  if (!fin.good()) {
    26	    cout << "File '" << file_name << "' not found!" << endl;
    27	    exit(1);
    28	  }
    29	  load(fin);
    30	}
    31	
    32	void digraph::load(istream &is) {
    33	  string line;
    34	  getline(is, line);
    35	  istringstream iss(line);
    36	  iss >> n;
    37	  resize(n);
    38	  while (!is.eof()) {
    39	    getline(is, line);
    40	    if (line.empty())
    41	      continue;
    42	    istringstream iss(line);
    43	    char c;
    44	    iss >> c;
    45	    int v, u;
    46	    if (c == 'l') {
    47	      iss >> v;
    48	      while (iss >> u)
    49	        add_edge(v, u);
    50	    } else if (c == 'p') {
    51	      iss >> v;
    52	      while (iss >> u) {
    53	        add_edge(v, u);
    54	        v = u;
    55	      }
    56	    }
    57	  }
    58	}
    59	
    60	int digraph::count_vertices() { return n; }
    61	
    62	bool digraph::is_edge(int a, int b) {
    63	  assert(0 <= a && a < n && 0 <= b && b < n);
    64	  return adj_mat[a][b];
    65	}
    66	
    67	void digraph::add_edge(int a, int b) {
    68	  assert(0 <= a && a < n && 0 <= b && b < n);
    69	  if (is_edge(a, b))
    70	    return;
    71	  input_vers[b].push_back(a);
    72	  adj_list[a].push_back(b);
    73	  adj_mat[a][b] = 1;
    74	}
    75	
    76	vector<pair<int, int>> digraph::edges() {
    77	  vector<pair<int, int>> es;
    78	  for (int v = 0; v < n; v++) {
    79	    for (int u : adj_list[v]) {
    80	      es.push_back(make_pair(v, u));
    81	    }
    82	  }
    83	  return es;
    84	}
    85	
    86	void digraph::print() {
    87	  cout << "Count of vertices: " << n << '\n';
    88	  cout << "Degree of vertices:\n";
    89	  for (int v = 0; v < n; v++) {
    90	    cout << "\tdeg(" << v << ") = " << degree(v) << endl;
    91	  }
    92	  cout << "Edges:\n";
    93	  for (int v = 0; v < n; v++) {
    94	    for (int u : adj_list[v]) {
    95	      cout << '\t' << v << " -> " << u << '\n';
    96	    }
    97	  }
    98	  if (is_leaves_finded) {
    99	    cout << "Leaves: ";
   100	    for (int i=0; i<leaves.size();i++) {
   101	      cout << leaves[i];
   102	      if (i < leaves.size() -1) cout << ", ";
   103	    }
   104	    cout << endl;
   105	  }
   106	  if (is_dists_finded) {
   107	    cout << "Dists:\n";
   108	    for (int v = 0; v < n; v++) {
   109	      cout << '\t';
   110	      for (int u = 0; u < n; u++) {
   111	        cout << dist[v][u];
   112	        if (u < n - 1)
   113	          cout << " ";
   114	      }
   115	      cout << '\n';
   116	    }
   117	  }
   118	  if (is_inv3_finded) {
   119	    cout << "Inv3:\n";
   120	    for (int v = 0; v < n; v++) {
   121	      cout << '\t' << v << ": '" << inv3[v] << "'\n";
   122	    }
   123	    cout << "full: '" << inv3_full << "'\n";
   124	  }
   125	}
   126	
   127	int digraph::degree(int v) { return adj_list[v].size(); }
   128	
   129	vector<int> digraph::output_vertices(int v) { return adj_list[v]; }
   130	
   131	void digraph::find_dists_from(int s) {
   132	  queue<int> q;
   133	  vector<bool> used(n, false);
   134	  q.push(s);
   135	  dist[s][s] = 0;
   136	  while (!q.empty()) {
   137	    int v = q.front();
   138	    q.pop();
   139	    used[v] = true;
   140	    for (int u = 0; u < n; u++) {
   141	      if (is_edge(v, u) || is_edge(u, v)) {
   142	        if (!used[u]) {
   143	          dist[s][u] = dist[s][v] + 1;
   144	          q.push(u);
   145	        }
   146	      }
   147	    }
   148	  }
   149	}
   150	
   151	void digraph::find_dists() {
   152	  if (is_dists_finded)
   153	    return;
   154	  for (int s = 0; s < n; s++) {
   155	    find_dists_from(s);
   156	  }
   157	  is_dists_finded = true;
   158	}
   159	
   160	void digraph::find_leaves() {
   161	  if (is_leaves_finded)
   162	    return;
   163	  for (int v = 0; v < n; v++) {
   164	    if (adj_list[v].size() == 0) {
   165	      leaves.push_back(v);
   166	    }
   167	  }
   168	  is_leaves_finded = true;
   169	}
   170	
   171	void digraph::find_inv3() {
   172	  find_dists();
   173	  find_leaves();
   174	  for (int v = 0; v < n; v++) {
   175	    stringstream ss;
   176	    ss << dist[root][v] << ", " << input_vers[v].size() << ", "
   177	       << adj_list[v].size() << ", (";
   178	    vector<int> t;
   179	    for (int i = 0; i < leaves.size(); i++) {
   180	      t.push_back(dist[v][leaves[i]]);
   181	    }
   182	    sort(t.begin(), t.end());
   183	    for (int i = 0; i < t.size(); i++) {
   184	      ss << t[i];
   185	      if (i < t.size() - 1)
   186	        ss << ", ";
   187	    }
   188	    ss << ")";
   189	    inv3[v] = ss.str();
   190	  }
   191	  vector<string> t = inv3;
   192	  sort(t.begin(), t.end());
   193	  for (int v = 0; v < n; v++) {
   194	    inv3_full += t[v];
   195	    if (v < n - 1)
   196	      inv3_full += "; ";
   197	  }
   198	  is_inv3_finded = true;
   199	}
\end{verbatim}
