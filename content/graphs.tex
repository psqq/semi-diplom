\chapter{Графы и их инварианты}

\section{Определения}

\begin{my_def}
Граф, или неориентированный граф $G$ — это упорядоченная пара $G:=(V,E)$, где $V$ — это непустое множество вершин или узлов, а $E$ — множество пар (в случае неориентированного графа — неупорядоченных) вершин, называемых рёбрами.
\end{my_def}

\begin{my_def}
Ориентированный граф (сокращённо орграф)  $G$ — это упорядоченная пара  $G:=(V,A)$, где  $V$ — непустое множество вершин или узлов, и  $A$ — множество (упорядоченных) пар различных вершин, называемых дугами или ориентированными рёбрами.
\end{my_def}

\begin{my_def}
Инвариант графа в теории графов — некоторое обычно числовое значение или упорядоченный набор значений (хэш-функция), характеризующее структуру графа  $G=\langle A,V\rangle$  и не зависящее от способа обозначения вершин или графического изображения графа. Играет важную роль при проверке изоморфизма графов, а также в задачах компьютерной химии.
\end{my_def}

\section{Примеры инвариантов}

К инвариантам графа относятся:

\begin{itemize}
\item Диаметр графа  $\mathrm {diam}$ (G) — длина кратчайшего пути (расстояние) между парой наиболее удаленных вершин.
\item Инвариант Колен де Вердьера.
\item Индекс Винера — величина  $w=\sum _{{\forall i,j}}d(v_{i},v_{j})$, где  $d(v_{i},v_{j})$ — минимальное расстояние между вершинами  $v_{i}$ и  $v_{j}$.
\item Индекс Рандича — величина  $r=\sum _{{(v_{i},v_{j})\in V}}{\frac  {1}{{\sqrt  {d(v_{i})d(v_{j})}}}}$.
\item Индекс Хосойи — число паросочетаний ребер графа плюс единица.
\item Мини-  $\mu _{{min}}(G)$ и макси-код  $\mu _{{max}}(G)$ матрицы смежности, получаемые путём выписывания двоичных значений матрицы смежности в строчку с последующим переводом полученного двоичного числа в десятичную форму. Мини-коду соответствует такой порядок следования строк и столбцов, при котором полученное значение является минимально возможным; макси-коду — соответственно максимальным.
\item Минимальное число вершин, которое необходимо удалить для получения несвязного графа.
\item Минимальное число ребер, которое необходимо удалить для получения несвязного графа.
\item Минимальное число вершин, необходимое для покрытия ребер.
\item Минимальное число ребер, необходимое для покрытия вершин.
\item Неплотность графа  $\epsilon (G)$ — число вершин максимального по включению безреберного подграфа (наибольшее количество попарно несмежных вершин). Несложно заметить, что  $\varphi (G)=\epsilon (\overline {G})$ и  $\epsilon (G)=\varphi (\overline {G})$.
\item Обхват графа — число ребер в составе минимального цикла.
\item Определитель матрицы смежности.
\item Плотность графа  $\varphi (G)$ — число вершин максимальной по включению клики.
\item Упорядоченный по возрастанию или убыванию вектор степеней вершин  $s(G)=(d(v_{1}),d(v_{2}),\dots ,d(v_{n}))$ — при использовании переборных алгоритмов определения изоморфизма графов в качестве предположительно-изоморфных пар вершин выбираются вершины с совпадающими степенями, что способствует снижению трудоемкость перебора. Использование данного инварианта для $k$-однородных графов не приводит к снижению вычислительной сложность перебора, так как степени всех вершин подобного графа совпадают:  $s(G)=(k,k,\dots ,k)$.
\item Упорядоченный по возрастанию или убыванию вектор собственных чисел матрицы смежности графа (спектр графа). Сама по себе матрица смежности не является инвариантом, так как при смене нумерации вершин она претерпевает перестановку строк и столбцов.
\item Число вершин  $n(G)=|A|$ и число дуг/ребер  $m(G)=|V|$.
\item Число компонент связности графа  $\kappa (G)$.
\item Число Хардвигера  $\eta (G)$.
\item Характеристический многочлен матрицы смежности.
\item Хроматическое число  $\chi (G)$.
\end{itemize}
