\begin{verbatim}
     1	#ifndef GRAPH_H
     2	#define GRAPH_H
     3	
     4	#include <bits/stdc++.h>
     5	using namespace std;
     6	
     7	class digraph {
     8	  int n = 0; // количество вершин
     9	  int root = 0; // корневая вершина
    10	  vector<vector<int>> adj_list; // списки смежных вершин
    11	  vector<vector<int>> adj_mat; // матрица смежности
    12	  vector<vector<int>> input_vers;
    13	  // Информация о графе полученная с помощью bfs алгоритма
    14	  // расстояние от одной вершины до другой
    15	  // без учета направлений ребер
    16	  bool is_dists_finded = false;
    17	  vector<vector<int>> dist;
    18	  bool is_leaves_finded = false;
    19	  vector<int> leaves;
    20	  // Инвариант 3
    21	  bool is_inv3_finded = false;
    22	  vector<string> inv3;
    23	  string inv3_full;
    24	private:
    25	  void resize(int new_n);
    26	  void add_edge(int a, int b);
    27	  void load(istream &is);
    28	  void find_dists_from(int v);
    29	public:
    30	  digraph(istream &is);
    31	  digraph(string s);
    32	  digraph(int an = 0);
    33	  void load_from_file(string file_name);
    34	  int count_vertices();
    35	  bool is_edge(int a, int b);
    36	  vector<pair<int, int>> edges();
    37	  void print();
    38	  int degree(int v);
    39	  vector<int> output_vertices(int k);
    40	  void find_dists();
    41	  void find_leaves();
    42	  void find_inv3();
    43	};
    44	
    45	#endif /* end of include guard: GRAPH_H */
\end{verbatim}
