\chapter{Выдержки из Википедии}
\section{Название}

\TeX{} (название произносится как «тех», от греч. $\tau\varepsilon\chi\nu\eta$ — <<искусство>>,
<<мастерство>>)   — система компьютерной вёрстки, разработанная американским профессором информатики
\href{http://en.wikipedia.org/wiki/Donald_Knuth}{Дональдом Кнутом} в целях создания компьютерной типографии. 
В неё входят средства для секционирования документов, для работы с перекрёстными ссылками. 

Многие считают TeX лучшим способом для набора сложных математических формул. В частности, из-за этих
возможностей, TeX популярен в академических кругах, особенно среди математиков и физиков.

\section{Дистрибутивы}

Распространённые комплекты вёрстки на основе ТеХ’а: 

\begin{itemize}
\item TeX Live
\item MikTeX
\item MacTeX
\end{itemize}