\begin{verbatim}
     1	#include <bits/stdc++.h>
     2	#include "digraph.h"
     3	using namespace std;
     4	
     5	class DigraphIso {
     6	  digraph g1, g2;
     7	  set<int> vertices;
     8	  int n = -1; // количество вершин в графах g1, g2
     9	  int g1_k = -1; // подмножество вершин g1 {0, 1, ..., g1_k}
    10	  set<int> g2_s; // Подмножество вершин графа g2, которое
    11	  // соотвествует множеству вершин грфафа g1 {0, 1, ..., g1_k}
    12	  vector<int> f; // биекция: i (вершина g1) -> f[i] (вершина g2)
    13	  int ans = -1;
    14	  bool match(int v);
    15	  void go();
    16	public:
    17	  DigraphIso(digraph ag1, digraph ag2);
    18	  bool is_iso();
    19	  vector<int> biection();
    20	};
    21	
    22	DigraphIso::DigraphIso(digraph ag1, digraph ag2)
    23	  : g1(ag1), g2(ag2) {}
    24	
    25	bool DigraphIso::match(int v) {
    26	  // Подмножество g1 {0, 1, ..., g1_k - 1} изоморфно
    27	  // подножеству графа g2 g2_s,
    28	  // которое равно {f[0], f[1], ..., f[g1_k - 1]}
    29	  // На данном шаге нужно проверить может ли подмножество g1
    30	  // {0, 1, ..., g1_k - 1, g1_k} быть изоморфно
    31	  // подмножеству g2 {f[0], f[1], ..., f[g1_k - 1], v}
    32	  for (int u = 0; u < g1_k; u++) {
    33	    if (g1.is_edge(u, g1_k) && !g2.is_edge(f[u], v)) {
    34	      // cout << 1 << ": " << u << " " << g1_k << " "
    35	      //      << f[u] << " " << v << endl;
    36	      return false;
    37	    }
    38	    if (g1.is_edge(g1_k, u) && !g2.is_edge(v, f[u])) {
    39	      // cout << 2 << endl;
    40	      return false;
    41	    }
    42	  }
    43	  return true;
    44	}
    45	
    46	void DigraphIso::go() {
    47	  // cout << "go begin\n";
    48	  // cout << "g2_s: ";
    49	  // for (int i : g2_s) cout << i << " ";
    50	  // cout << endl;
    51	  if (g2_s.size() == n) {
    52	    ans = 1;
    53	    return;
    54	  }
    55	  g1_k += 1;
    56	  set<int> t;
    57	  set_difference(
    58	    vertices.begin(), vertices.end(),
    59	    g2_s.begin(), g2_s.end(),
    60	    inserter(t, t.end())
    61	  );
    62	  // cout << "t: ";
    63	  // for (int i : t) cout << i << " ";
    64	  // cout << endl;
    65	  for (int v : t) {
    66	    if (ans > 0) return;
    67	    if (match(v)) {
    68	      // cout << g1_k << " match " << v << endl;
    69	      f[g1_k] = v;
    70	      g2_s.insert(v);
    71	      go();
    72	      g2_s.erase(v);
    73	    } else {
    74	      // cout << g1_k << " don't match " << v << endl;
    75	    }
    76	  }
    77	  g1_k -= 1;
    78	}
    79	
    80	bool DigraphIso::is_iso() {
    81	  if (ans >= 0) return ans;
    82	  if (g1.count_vertices() != g2.count_vertices()) {
    83	    return false;
    84	  }
    85	  n = g1.count_vertices();
    86	  f.resize(n);
    87	  for (int i=0; i<n; i++) vertices.insert(i);
    88	  go();
    89	  cout << "ans " << ans << endl;
    90	  if (ans < 0) ans = 0;
    91	  return ans;
    92	}
    93	
    94	vector<int> DigraphIso::biection() {
    95	  return f;
    96	}
    97	
    98	int main(int argc, char **argv) {
    99	  if (argc < 3) {
   100	    cout << "few arguments" << endl;
   101	    return 1;
   102	  }
   103	  if (argc > 1 && string(argv[1]) == "-i") {
   104	    digraph g;
   105	    g.load_from_file(argv[2]);
   106	    g.find_inv3();
   107	    g.print();
   108	    return 0;
   109	  }
   110	  digraph g1, g2;
   111	  g1.load_from_file(argv[1]);
   112	  g2.load_from_file(argv[2]);
   113	  cout << "# 1.digraph:\n";
   114	  g1.print();
   115	  cout << "# 2.digraph:\n";
   116	  g2.print();
   117	  DigraphIso iso(g1, g2);
   118	  cout << "Is isomorph?: " << iso.is_iso() << endl;
   119	  if (iso.is_iso()) {
   120	    vector<int> f = iso.biection();
   121	    cout << "biection:\n";
   122	    for (int i=0; i<g1.count_vertices(); i++) {
   123	      cout << '\t' << i << " -> " << f[i] << '\n';
   124	    }
   125	  }
   126	}
\end{verbatim}
