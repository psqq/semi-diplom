\chapter{Полугруппы и полурешетки}

\begin{my_def}
Бинарная опрация~$*$ на множестве~$S$ --- это функция
$$f: S^2 \rightarrow S.$$
Значение бинарной операции для $x, y \in S$ обозначают~$x * y$.
\end{my_def}


\begin{my_def}
Пусть~$*$ --- бинарная операция определённая на множесте~$S$, а
$x, y, z \in S$~--- произвольные элементы множества $S$. Тогда если
выполняется равенсто
$$x * y = y * x,$$
то операция~$*$ называется коммутативной.
Если выполняется
$$
x * (y * z) = (x * y) * z,
$$
то операция~$*$ называется ассоциативной.
Если выполняется
$$
x * x = x,
$$
то операция~$*$ называется идемпотентной.
\end{my_def}


\begin{my_def}
Полугруппа --- множество с заданной на нём ассоциативной бинарной операцией~$(S,*)$.
\end{my_def}


\begin{my_def}
Полурешетка --- это полургуппа, бинарная операция которой коммутативна и
идемпотентна.
\end{my_def}

Для полрешетки можно определить частичный порядок

$$x \leq y \Leftrightarrow x*y=x.$$
